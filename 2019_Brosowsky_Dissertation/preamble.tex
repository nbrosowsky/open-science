\usepackage{xspace}
\newcommand{\yeardegree}{2019\xspace}\newcommand{\degree}{Doctor of Philosophy\xspace}
\newcommand{\field}{Psychology\xspace}
\newcommand{\chairperson}{Matthew J.C. Crump, Ph.D.\xspace}
\newcommand{\committeeone}{Andrew R. Delamater, Ph.D.\xspace}
\newcommand{\committeetwo}{Timothy J. Ricker, Ph.D.\xspace}
\newcommand{\committeethree}{Julie M. Bugg, Ph.D.\xspace}
\newcommand{\gradschoolguy}{\xspace}
\newcommand{\EO}{Richard Bodnar, Ph.D.\xspace}
\newcommand{\advisor}{Matthew J.C. Crump, Ph.D\xspace}
\newcommand{\abstract}{Cognitive control enables flexible goal-directed behavior via attention and action selection processes that prioritize goal-relevant over irrelevant information. These processes allow us to behave flexibly in the face of contradicting or ambiguous information and update behavior in response to the changing environment. Furthermore, they are thought to be in direct opposition to learned, automatic processing in that they enable us to disregard learned behaviors when they are inconsistent with our current goals. The strict dichotomy between stimulus-driven and goal-driven influences, however, has downplayed the role of memory in guiding attention. The position forwarded in this thesis is that a memory-based framework is needed to fully understand attentional control. People often re-encounter similar objects, tasks, and environments that require similar cognitive control operations. A memory-retrieval process could shortcut the slow, effortful, and resource-demanding task of updating control settings by retrieving and reinstating the control procedures used in the past. The aim of the current thesis is to empirically test general principles of an instance theory of automatic attentional control using a converging operations approach. In Chapter 2, I examine the obligatory nature of memory encoding by investigating context-specific proportion congruent effects in a non-conflict selective attention task. In Chapter 3, I examine the assumption of long-term instance-based representation by investigating long-term single-trial effects in a context-cuing flanker paradigm. Finally, in Chapter 4 I examine how memory retrieval can influence context-specific attentional control in a context-specific proportion congruent task.\xspace}
% Tables
      \usepackage{booktabs}
      \usepackage{threeparttable}
      \usepackage{array}
      \newcolumntype{x}[1]{%
      >{\centering\arraybackslash}m{#1}}%
      \usepackage{placeins}
      \usepackage{chngcntr}
      \counterwithin{figure}{chapter}
      \counterwithin{table}{chapter}
      \usepackage[makeroom]{cancel}
